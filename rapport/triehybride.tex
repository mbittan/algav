\documentclass[10pt]{report}

\usepackage[utf8]{inputenc} 
\usepackage[francais]{babel}
\usepackage{geometry} 
\geometry{a4paper} 
\usepackage{booktabs} 
\usepackage{graphicx} 
\usepackage{algorithm}
\usepackage{algpseudocode} 
\usepackage{pgf}
\renewcommand{\algorithmicend}{\textbf{fin}}
\renewcommand{\algorithmicif}{\textbf{si}}
\renewcommand{\algorithmicthen}{\textbf{alors}}
\renewcommand{\algorithmicelse}{\textbf{sinon}}
\newcommand{\algorithmicelsif}{\algorithmicelse\ \algorithmicif}
\newcommand{\algorithmicendif}{\algorithmicend\ \algorithmicif}
\renewcommand{\algorithmicreturn}{\textbf{retourner}}
\renewcommand{\algorithmicfunction}{\textbf{fonction}}
\begin{document}

\begin{center}
  \begin{table}
    \begin{tabular}{|l|l|l|}
      \hline  \textbf{Fonctions} & \textbf{Emplacement} & \textbf{Complexité}\\  \hline
      \textbf{trie\_hybride} & src/trieHybride\_primitives.c, $l4-17$ & $\Theta(1)$ \\ \hline 
      \multicolumn{3}{|l|}{\it{Fonction permettant de créer un trie hybride.}} \\ \hline
      \textbf{free\_trie\_hybride} & src/trieHybride\_primitives.c, $l20-27$ & $\Theta(n)$ \\ \hline 
      \multicolumn{3}{|l|}{\it{Libère la mémoire associée au trie hybride.}} \\ \hline
      \textbf{racine} & src/trieHybride\_primitives.c, $l30-32$  & $\Theta(1)$ \\ \hline 
      \multicolumn{3}{|l|}{\it{Renvoie le caractère présent à la racine du trie hybride.}} \\ \hline
      \textbf{inferieur} & src/trieHybride\_primitives.c, $l35-37$  & $\Theta(1)$ \\ \hline 
      \multicolumn{3}{|l|}{\it{Renvoie le sous arbre qui représente tout les mots qui commencent par une}} \\     
      \multicolumn{3}{|l|}{\it{lettre inférieure à celle présente à la racine.}} \\ \hline  
      \textbf{egal} & src/trieHybride\_primitives.c, $l39-41$  & $\Theta(1)$ \\ \hline 
      \multicolumn{3}{|l|}{\it{Renvoie le sous arbre qui représente tout les mots qui commencent par la}} \\ 
      \multicolumn{3}{|l|}{\it{lettre qui se trouve à la racine.}} \\ \hline  
      \textbf{superieur} & src/trieHybride\_primitives.c, $l43-45$  & $\Theta(1)$ \\ \hline 
      \multicolumn{3}{|l|}{\it{Renvoie le sous arbre qui représente tout les mots qui commencent par une}} \\     
      \multicolumn{3}{|l|}{\it{lettre supérieure à celle présente à la racine.}} \\ \hline  
      \textbf{est\_trie\_vide} & src/trieHybride\_primitives.c, $l48-50$  & $\Theta(1)$ \\ \hline 
      \multicolumn{3}{|l|}{\it{Renvoie vrai si le trie est vide}} \\   \hline   
      \textbf{creer\_mot} & src/trieHybride\_primitives.c, $l53-66$  & $\Theta(lg(mot))$ \\ \hline 
      \multicolumn{3}{|l|}{\it{Crée le trie hybride représentant le mot passé en paramètre.}} \\  \hline
      \textbf{ajouter\_trie\_hybride} & src/trieHybride\_primitives.c, $l175-205$  & $\Theta(log(n)+L)$ \\ \hline 
      \textbf{recherche\_trie\_hybride} & src/trieHybride\_simple.c, $l7-36$  & $\Theta(log(n)+L)$ \\ \hline 
      \textbf{comptage\_mots} & src/trieHybride\_simple.c, $l39-52$  & $\Theta(n)$ \\ \hline 
      \textbf{afficher\_liste\_mots} & src/trieHybride\_simple.c, $l58-76$  & $\Theta(n)$ \\ \hline 
      \textbf{liste\_mots} & src/trieHybride\_simple.c, $l81-101$  & $\Theta(n)$ \\ \hline 
      \textbf{comptage\_nil} & src/trieHybride\_simple.c, $l106-116$  & $\Theta(n)$ \\ \hline 
      \textbf{hauteur} & src/trieHybride\_simple.c, $l130-140$  & $\Theta(n)$ \\ \hline 
      \textbf{profondeur\_moyenne} & src/trieHybride\_simple.c, $l145-154$  & $\Theta(n)$ \\ \hline 
      \textbf{prefixe} & src/trieHybride\_simple.c, $l159-187$  & $O(n)$ \\ \hline 
      \textbf{supprimer} & src/trieHybride\_simple.c, $l257-297$  & $\Theta(log(n))$ \\ \hline
      \textbf{conversion} & src/trieHybride\_complexe.c, $l4-49$  & $\Theta(n)$ \\ \hline  
      \textbf{equilibrer} & src/trieHybride\_complexe.c, $l70-101$  & $O(n^2)$ \\ \hline  
    \end{tabular}
  \end{table}
\end{center}

Explication des complexités ($n$ représente le nombre de clés insérées):
\begin{itemize}
\item Insertion (pareil pour recherche): Si le trie est plus ou moins equilibré, on divise par trois l'espace dans lequel on doit chercher le mot à chaque appel récursif. On a donc l'équation de réccurence suivante : \\
  \begin{center}
    $T(n)=T(n/3)+\Theta(1)$
  \end{center}
  D'après la deuxième règle du theorème Maitre ($a=1,b=3$), on a $T(n)=log_3(n)$. Sachant que l'on parcourt forcément toute la chaine qui représente le mot, il faut prendre ceci en compte dans le calcul de la complexité. On a donc   la complexité suivante : $log_3(n)+L$, avec $L$ la longueur du mot que l'on cherche (ou insère).\\
  Dans le cas ou le trie est totalement déséquillibré, on a une complexité proche de celle d'un arbre de la briandais. En effet, le trie hybride le plus déséquilibré est le trie qui n'a pas de fils gauche (ou droit). C'est le cas ou tout les mots ont été insérés dans l'ordre alphabetique(ou l'inverse de l'ordre alphabetique). Chaque noeud à alors deux fils, le premier représentant les mots commencant par la lettre contenue dans le noeud, et le second représentant les mots commençant par une autre lettre.
\item Comptage\_mots (pareil pour afficher\_liste\_mots, liste\_mots, hauteur, profondeur\_moyenne, comptage\_nil) : Pour ces fonctions, dans tout les cas on parcourt l'ensemble de l'arbre. Si l'arbre est equilibré, on a l'équation de récurrence suivante :
  \begin{center}
    $T(n)=3T(n/3)+\Theta(1)$
  \end{center} 
  D'après le cas 1 du théoreme Maitre, le resultat de cette équation est : $T(n)=n$\up{$log_33$}$=n $ avec $a=3,b=3$ et $\epsilon=1$.
\end{itemize}

\section{Suppression} 
Principe : On parcourt d'abord le trie pour trouver le noeud qui correspond a la fin du mot que l'on veut supprimer.
Si le mot n'est pas présent, le trie reste inchangé. Une fois que l'on a trouvé ce noeud, on indique que celui-ci ne représente plus la fin d'un mot. Si ce noeud ne possede pas de fils $egal$, cela veut dire qu'il n'y a aucun mot dans l'arbre prefixé par celui que l'on veut supprimer. On peut donc supprimer ce noeud. Trois cas peuvent alors se produire : \\
\begin{enumerate}
\item Le noeud n'a aucun fils. On peut alors le supprimer sans se soucier de ses fils. 
\item Le noeud a un seul fils (soit le gauche, soit le droit). Dans ce cas, peut remplacer le noeud à supprimer par son fils.
\item Le noeud a deux fils (gauche et droit donc). Il faut alors trouver un remplaçant pour le noeud que l'on va supprimer. Le bon candidat est le minimum (le fils le plus a gauche, qui n'a pas de fils gauche) du fils droit, comme on le ferait dans un ABR. En effet, dans ce cas, on a un noeud qui contient un caractere plus grand que tout les elements du sous-arbre gauche, et plus petit que tout ceux du sous-arbre droit. Cela nous permet de garder les proprietes du trie hybride. De plus, comme le minimum n'a pas de fils gauche, on peut le remplacer par son fils droit sans problème.
\end{enumerate}
Une fois le noeud supprimé, il est possible que l'on doive supprimer le père de celui-ci. On effectue la suppression si la branche du milieu du père est vide, et que le père ne représente pas la fin d'un mot.
Code : \\
\begin{algorithm}
  \caption{Suppression}
  \begin{algorithmic}[1]
    \Function{Suppression}{$T,mot$}
    \If {$T$ est vide}
    \State {\textbf{retourner}} $T$
    \EndIf
    \If{$lg(mot)=1$ \textbf{et} $premier(mot)=racine(T)$}\Comment{Noeud de la fin du mot}
    \State $T.fin \gets \textbf{faux}$ \Comment{Le noeud ne représente plus la fin du mot} 
    \If{$egal(T)$ est vide}
    \State {\textbf{retourner}} $supprimer\_racine(T)$
    \Else
    \State {\textbf{retourner}} $T$
    \EndIf
    \ElsIf {$premier(mot)<racine(T)$}
    \State $T.inferieur=suppression(T.inferieur,mot)$
    \ElsIf {$premier(mot)>racine(T)$}
    \State $T.superieur=suppression(T.superieur,mot)$
    \Else
    \State $T.egal=suppression(T.egal,reste(mot))$
    \If {$T.egal$ est vide \textbf{et} $T.fin=\textbf{faux}$}
    \State {\textbf{retourner}} $supprimer\_racine(T)$
    \Else
    \State {\textbf{retourner}} $T$
    \EndIf
    \EndIf
    \State {\textbf{retourner}} $T$
    \EndFunction
  \end{algorithmic}
\end{algorithm}

La fonction supprimer\_racine est la fonction qui permet de gérer les trois cas énoncés précédemment.

\begin{algorithm}
  \caption{supprimer\_racine}
  \begin{algorithmic}[1]
    \Function{supprimer\_racine}{$T$}
    \If {$T.inferieur$ est vide \textbf{et} $T.superieur$ est vide}
    \State {\textbf{liberer}} $T$
    \State {\textbf{retourner}} $Trie$ $vide$
    \ElsIf {$T.inferieur$ est vide}
    \State $T \gets T.superieur$ \Comment{On met l'arbre de droite a la place de T}
    \ElsIf {$T.superieur$ est vide}
    \State $T \gets T.inferieur$ \Comment{On met l'arbre de gauche a la place de T}
    \Else \Comment{Les sous-arbre droit et gauche ne sont pas vide}
    \State $(T.superieur,min) \gets extraire\_min(T.superieur)$
    \State $T \gets min$
    \EndIf
    \State {\textbf{retourner}} $T$
    \EndFunction
  \end{algorithmic}
\end{algorithm}

La fonction extraire\_min est une fonction qui permet de recuperer le minimum d'un trie. Le pere du minimum reçoit alors le fils droit de celui. Le minimum que l'on recupère avec cette fonction n'a donc ni fils gauche, ni fils droit.

\section{Equilibrage} 
Principe : La fonction d'équilibrage fonctionne dans le même esprit que celle des AVL. Dans un premier temps on appelle récursivement la fonction sur les trois sous-arbres. Ensuite, on calcule la difference de hauteur entre le sous-arbre gauche et le sous-arbre droit. Si cette différence est plus grande que deux, on effectue une rotation droite de l'arbre. De plus, si les deux fils du fils gauche ont une difference de hauteur plus grande (en valeur absolue) que -1, on effectue une rotation gauche sur ce fils. Si la différence de hauteur est plus petite que -2, on effectue le symétrique des opérations précédentes (i.e. les rotations gauche deviennent des rotations droites et inversement).
\begin{algorithm}
  \caption{equilibrage}
  \begin{algorithmic}[1]
    \Function{equilibrage}{$T$}
    \If {$T$ est vide}
    \State {\textbf{retourner}} $T$
    \EndIf
    \State $T.inferieur \gets equilibrage(T.inferieur)$
    \State $T.egal \gets equilibrage(T.egal)$
    \State $T.superieur \gets equilibrage(T.superieur)$
    \State $diff \gets hauteur(t.inferieur)-hauteur(t.superieur)$
    \If {$diff >= 2$}   
    \If {$hauteur(t.inferieur.inferieur)-hauteur(t.inferieur.superieur) <= -1$}
    \State $T.inferieur=rotG(T.inferieur)$
    \EndIf  
    \State $T=rotD(T)$
    \ElsIf {$diff <= -2$}   
    \If {$hauteur(t.superieur.inferieur)-hauteur(t.superieur.superieur) >= 1$}
    \State $T.superieur=rotD(T.superieur)$
    \EndIf  
    \State $T=rotG(T)$
    \EndIf
    \State {\textbf{retourner}} $T$
    \EndFunction
  \end{algorithmic}
\end{algorithm}

Les fonctions de rotation sont quasiment identiques a celles présentées en cours. Leur complexité est donc $\Theta(1)$. La complexité de la fonction d'équilibrage est $O(n^2)$, car pour chaque noeud on calcule la hauteur de ses filsdroit et gauche. On pourrait améliorer la complexité en ajoutant un champs $hauteur$ dans la structure, tenu à jour a chaque insertion/suppression.
\end{document}
